%%%%%%%%%%%%%%%
% Definitions %
%%%%%%%%%%%%%%%

\section{Definitions}

%%%%%%%%%%%%%%%%%%%%%%%%%%%%%%
% Basic Notions and Notation %
%%%%%%%%%%%%%%%%%%%%%%%%%%%%%%

\subsection{Basic Notions and Notation}

\textit{In the following, $\Omega$ is a set, $\CalF$ a \SigmaAlgebra\ on $\Omega$. If used, then $\mu$ is a measure. Otherwise, the measure is the probability measure $P$.}

\begin{definition}{1.1}{}

    Let $\mathcal{F}$ be a family of subsets of set $\Omega$. $\mathcal{F}$ is called a \emph{$\sigma$-algebra} if:

        \begin{itemize}
            \setlength{\parskip}{0em}
            \item \emph{Closed Under Complement}: $A \in \mathcal{F} \Rightarrow A^c \in \mathcal{F}$,
            \item \emph{Closed Under Arbitrary Union}: $A_n \in \mathcal{F}$ for integer $n \geq 1$ $\Rightarrow \bigcup_{n=1}^{\infty}A_n \in \mathcal{F}$,
            \item \emph{Contains Entire Set}: $\Omega \in \mathcal{F}$
        \end{itemize}

\end{definition}

\begin{definition}{1.2}{}
    Let $\mathcal{C}$ be a family of subsets of $\Omega$. There exists a $\sigma$-algebra which contains $\mathcal{C}$ \emph{and} which is contained in every $\sigma$-algebra that contains $\mathcal{C}$ (take intersection of all $\sigma$-algebras. Such $\sigma$-algebra is \emph{unique} and called \emph{smallest $\sigma$-algebra containing $\mathcal{C}$} or \emph{$\sigma$-algebra generated by $\mathcal{C}$}, denoted by $\sigma(\mathcal{C})$. Simplest example, let $A \subseteq \Omega$:

        \begin{align*}
            \sigma(A) = \{\emptyset, A, A^c, \Omega\}.
        \end{align*}
\end{definition}

\begin{definition}{}{Finite Measure Space}

    Let $(\Omega, \CalF, \mu)$ be a measure space. If $\mu(\Omega) < \infty$, then we call the measure space \emph{finite}.

\end{definition}

%%%%%%%%%%%%%%%%%%%%
% Random Variables %
%%%%%%%%%%%%%%%%%%%%

\subsection{Random Variables}

\begin{definition}{2.1.1}{}

    Let $A \subseteq \Omega$ and $\Indicator{A}$ be defined as follows:

        \begin{align*}
            \Indicator{A}(\omega) =
            \begin{cases}
                1, & \omega \in A \\
                0, & \omega \not\in A
            \end{cases}.
        \end{align*}

    Then $\Indicator{A}$ is a R.V. and called the \emph{indicator (function) of (events) A}.

\end{definition}

\begin{definition}{2.3}{Distribution Function}

    Let $X$ be a random variable. Then the function

        \begin{align*}
            F_X(x) &= P(X \leq x) = \\ &= P(X \in (-\infty, x]) = Q_X((-\infty, x]),
        \end{align*}

    for $x \in \mathbb{R}$ is called the \emph{distribution function} of $X$.

\end{definition}

%%%%%%%%%%%%%%%%%%%%%%%%%
% Expectation Integrals %
%%%%%%%%%%%%%%%%%%%%%%%%%

\subsection{Expectation Integrals}

\begin{definition}{}{Indicator Integral}

    Let $A \subseteq \Omega$, then:

        \begin{align*}
            \int_{\Omega} \Indicator{A} \,d\mu = \mu(A).
        \end{align*}

\end{definition}

\begin{definition}{}{Simple Function}

    Let $f: \Omega \to \mathbb{R}$ be a \emph{simple function}, then $f$ takes finitely many values. Formally, if $I$ is a finite index set, $(A_i)_{i \in I}$ a famility of \emph{disjoint} subsets of $\Omega$ and $(c_i)_{i \in I}$ a family of real numbers, then:

        \begin{align*}
            f(\omega) = \sum_{i \in I} c_i \Indicator{A_i}(\omega).
        \end{align*}

\end{definition}

\begin{definition}{}{Lebesgue Integral for Expectation}

    Let $X$ be a random variable. Then we write:

        \begin{align*}
            EX = \int_{\Omega} X \,dP.
        \end{align*}

\end{definition}

\begin{definition}{}{Non-negative, Measurable Lebesgue Integral}

    Let $f: \Omega \to \overline{\mathbb{R}}$ be a \emph{non-negative}, measurable function and $(f_n)_{n=1}^{\infty}$ a sequence of \emph{non-negative, simple} functions sucht that $\lim_{n \to \infty} f_n = f$. Then

        \begin{align*}
            \int_{\Omega} f \, d\mu = \lim_{n \to \infty} f_n \, d\mu.
        \end{align*}

\end{definition}

\begin{definition}{}{Lebesgue Integral}

     Let $f: \Omega \to \overline{\mathbb{R}}$ be a measurable function. The \emph{Lebesgue Integral} of $f$ is defined as:

        \begin{align*}
            \int_{\Omega} f \,d\mu = \int_{\Omega} f^+ \,d\mu - \int_{\Omega} f^- \,d\mu,
        \end{align*}

    where $f^+ = \max\{f, 0\}$ and $f^-=\max\{-f, 0\}$, if at least one of the integrals on the right-hand side is finite. If both are infinite, then we say that the Lebesgue Integral of $f$ does not exist.

\end{definition}

\begin{definition}{}{Restricted Integration}

    Let $A \in \CalF$ and $f: \Omega \to \overline{\mathbb{R}}$ is a measurable function, then we define:

        \begin{align*}
            \int_A f \, d\mu = \int_{\Omega} \Indicator{A} f \, d\mu,
        \end{align*}

    when the integral of $\Indicator{A} f$ w.r.t $\mu$ exists.

\end{definition}

\begin{definition}{3.7}{Absolute Continuity}

    Let $\mu$ and $\nu$ be measures on \SigmaAlgebra\ $\CalF$ such that for some $\CalF$-measureable $g: \Omega \to \mathbb{R}$:

        \begin{align*}
            \nu(A) = \int_{\Omega} \Indicator{A} g \, d\mu = \int_{A} g \mu(dx),
        \end{align*}

    for all $A \in \CalF$. Then $\nu$ is called \emph{absolutely continuous} with respect to $\mu$ and $g$ is called the \emph{density} or \emph{Radon-Nikodym derivative} (Notation: $g = \frac{d\nu}{d\mu}$).

\end{definition}

%%%%%%%%%%%%%%%%%%%%%%%%%%%%%%%%%%%%%%%
% Convergence of Measurable Functions %
%%%%%%%%%%%%%%%%%%%%%%%%%%%%%%%%%%%%%%%

\subsection{Convergence of Measurable Functions}

\begin{definition}{}{$\mu$-Almost Everywhere Finite}

    Let $f: \Omega \to \overline{\mathbb{R}}$ be $\CalF$-measurable, then $f$ is said to be \emph{$\mu$-almost everywhere} ($\mu$-a.e.) finite if $\mu(|f| = \infty) = 0$.

\end{definition}

\begin{definition}{}{Almost Surely Finite}

    Let $f: \Omega \to \overline{\mathbb{R}}$ be $\CalF$-measurable, then $f$ is said to be \emph{almost surely} (a.s.) finite if $P(|f| = \infty) = 0$ $\Leftrightarrow P(|f| < \infty) = 1$.

\end{definition}

\begin{definition}{5.1}{$\mu$-Almost Everywhere Convergence}

    Let $(f_n)_{n=1}^{\infty}$ be $\CalF$-measurable functions. The $f_n$ are said to \emph{converge $\mu$-almost everywhere} to a \emph{$\mu$-a.e. finite} $f: \Omega \to \overline{\mathbb{R}}$ as $n \to \infty$ if there exists an $A \in \CalF$ s.t. $\mu(A) = 0$ and

        \begin{align*}
            \lim_{n \to \infty} f_n(\omega) = f(\omega) \in \mathbb{R}, \quad \forall \omega \in A^C.
        \end{align*}

    \emph{Notation:} $\lim_{n \to \infty} f_n = f$ ($\mu$-a.e.) or $f_n \to f$ ($\mu$-a.e.).

\end{definition}

\begin{definition}{5.1}{Almost Sure Convergence}

    Let $(f_n)_{n=1}^{\infty}$ be $\CalF$-measurable functions. The $f_n$ are said to \emph{converge almost surely} to a \emph{a.s. finite} $f: \Omega \to \overline{\mathbb{R}}$ as $n \to \infty$ if there exists an $A \in \CalF$ s.t. $P(A) = 0$ and

        \begin{align*}
            \lim_{n \to \infty} f_n(\omega) = f(\omega) \in \mathbb{R}, \quad \forall \omega \in A^C.
        \end{align*}

    \emph{Notation:} $\lim_{n \to \infty} f_n = f$ (a.s.) or $f_n \to f$ (a.s.).

\end{definition}

\begin{definition}{5.2}{Convergence in Measure}

    Let $(f_n)_{n=1}^{\infty}$ be $\CalF$-measurable functions. The $f_n$ are said to \emph{converge in measure $\mu$} to a \emph{$\mu$-a.e. finite} $f: \Omega \to \overline{\mathbb{R}}$ as $n \to \infty$ if

        \begin{align*}
            \lim_{n \to \infty} \mu(|f_n - f| \geq \varepsilon) = 0, \quad \forall \varepsilon > 0.
        \end{align*}

    \emph{Notation:} $\mu-\lim_{n \to \infty} f_n = f$.

\end{definition}

\begin{definition}{5.2}{Convergence in Probability}

    Let $(f_n)_{n=1}^{\infty}$ be $\CalF$-measurable functions. The $f_n$ are said to \emph{converge in probability} to a \emph{a.s. finite} $f: \Omega \to \overline{\mathbb{R}}$ as $n \to \infty$ if

        \begin{align*}
            \lim_{n \to \infty} P(|f_n - f| \geq \varepsilon) = 0, \quad \forall \varepsilon > 0.
        \end{align*}

\end{definition}

\begin{definition}{}{Bounded in Measure}

    Let $(f_n)_{n=1}^{\infty}$ be a sequence of measurable functions, then it is \emph{bounded in measure $\mu$} if

        \begin{align*}
            \lim_{K \to \infty} \mu(|f_n| \geq K) = 0,
        \end{align*}

    for any $n \geq 1$.

\end{definition}

\begin{definition}{}{Bounded Uniformly in Measure}

    Let $(f_n)_{n=1}^{\infty}$ be a sequence of measurable functions, then it is \emph{bounded in measure $\mu$, uniformly in $n$} if

        \begin{align*}
            \lim_{K \to \infty} \sup_{n \geq 1} \mu(|f_n| \geq K) = 0.
        \end{align*}

\end{definition}

\begin{definition}{}{Finite Second Moment}

    Let $X$ be a random variable. Then $X$ has \emph{finite second moment} if $EX^2 < \infty$.

\end{definition}

%%%%%%%%%%%%%%%%
% Independence %
%%%%%%%%%%%%%%%%

\subsection{Independence of Events and Random Variables}

\begin{definition}{6.5}{$\lambda$-system}

    Let $\Lambda$ be a family o subsets of $\Omega$. Then $\Lambda$ is a $\lambda$-system, if it satisfies all of the following properties:

        \begin{enumerate}[(i)]
            \item (Contains whole set) $\Omega \in \Lambda$;
            \item (Closed under Subset Set Subtraction) if $A, B \in \Lambda$, such that $B \subset A$, then $A \setminus B \in \Lambda$;
            \item (Closed under Disjoint Union) if $(A_n)_{n=1}^{\infty}$ is a \emph{pairwise disjoint} sequence, i.e. $A_i \cap A_j = \emptyset$ for $i \neq j$, of subsets, such that $A_i \in \Lambda$ for $i = 1,2,\hdots$, then $\bigcup_{n=1}^{\infty} \in \Lambda$.
        \end{enumerate}

\end{definition}

\begin{definition}{6.6}{$\pi$-system}

    Let $\Pi$ be a family of subsets of $\Omega$. Then $\Pi$ is a $\pi$-system, if it is closed under finite intersections, i.e. $A,B \in \Pi$ $\Rightarrow$ $A \cap B \in \Pi$.

\end{definition}

%%%%%%%%%%%%%%%%%%%%%%%%%%%
% Conditional Expectation %
%%%%%%%%%%%%%%%%%%%%%%%%%%%

\subsection{Conditional Expectation}

\begin{definition}{}{Sub-$\sigma$-Algebra Measurable}

    Let $Y$ be a random variable TODO FINISH

\end{definition}
