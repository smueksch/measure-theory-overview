%%%%%%%%%%%%%%%%%%%%%%%%%
% Expectation Integrals %
%%%%%%%%%%%%%%%%%%%%%%%%%

\section{Expectation Integrals}

\begin{proposition}{(Unknown)}{}

    Let $A,B \subseteq \Omega$. Then the following equalities hold:

        \begin{itemize}
            \setlength{\parskip}{0em}
            \item $\Indicator{A^C} = 1 - \Indicator{A}$,
            \item $\Indicator{A \cap B} = \Indicator{A}\Indicator{B}$.
            \item $\Indicator{A \cup B} = \Indicator{A} + \Indicator{B} - \Indicator{A \cap B}$.
        \end{itemize}

\end{proposition}

\begin{lemma}{3.3}{}

    Let $X$ be a \emph{non-negative} random variable. Then there exists a sequence of \emph{non-negative, simple} random variables $X_n$ converging to $X$ for every $\omega \in \Omega$.

    \Hint $h_n(x)=\min\{\floor{2^nx}/2^n, n\}$ is non-negative, simple and increasing, approaching $x$. Consider $X_n \coloneqq h(X) \to X$.

\end{lemma}

\begin{lemma}{}{Simple Function Integral Properties}

    Let $f,g: \Omega \to \overline{\mathbb{R}}$ be a \emph{non-negative}, simple functions and $a,b \geq 0$. Then the following holds:

    \begin{itemize}
        \setlength{\parskip}{0em}
        \item $\int_{\Omega} f \, d\mu \geq 0$,
        \item $\int_{\Omega} (af + bg) \, d\mu = a\int_{\Omega} f + b\int_{\Omega}g \, d\mu$.
    \end{itemize}

\end{lemma}

\begin{corollary}{}{Positive Integral over Set}

    Let $A \subseteq \Omega$ and $f: \Omega \to \overline{\mathbb{R}}$ a \emph{non-negative} measurable function. Then $\int_A f \, d\mu \geq 0$.

\end{corollary}

\begin{lemma}{3.3}{General}

    Let $f: \Omega \to \overline{\mathbb{R}}$ be a \emph{non-negative}, measurable function. The there exists a sequence $f_n$ of \emph{non-negative}, simple functions such that:

    \begin{align*}
        \lim_{n \to \infty} f_n = f
    \end{align*}

    \Hint Use $h_n$ from Lemma 3.3's hint.

\end{lemma}

\begin{exercise}{3.5}{}

    Let $A \in \CalF$ s.t. $\mu(A) = 0$. Then for \emph{any} measurable function $f: \Omega \to \overline{\mathbb{R}}$:
    
        \begin{align*}
            \int_A f \,d\mu = 0.
        \end{align*}

\end{exercise}

\begin{exercise}{3.6}{}

    Let $f: \Omega \to \mathbb{R}$ be a measurable function, then:

        \begin{enumerate}[(i)]
            \setlength{\parskip}{0em}
            \item For any $c \in \mathbb{R}$ and $A \in \CalF$:

                \begin{align*}
                    \int_A cf \, d\mu = c \int_A f \, d\mu,
                \end{align*}
            
            provided the integral exists.
            \item For any $A, B \in \CalF$, such that $A \cap B = \emptyset$:

                \begin{align*}
                    \int_{A \cup B} f \,d\mu = \int_{A} f \,d\mu + \int_{B} f \,d\mu,
                \end{align*}

            provided the left-hand or right-hand side is well-defined.
        \end{enumerate}

\end{exercise}

\begin{theorem}{3.8}{Monotone Convergence}

    Let $(f_n)_{n=1}^{\infty}$ be increasing sequence of non-negative, measurable functions $f_n: \Omega \to \overline{\mathbb{R}}$, converging to some $f$. Then:

        \begin{align*}
            \int_{\Omega} \lim_{n \to \infty} f_n \,d\mu = \lim_{n \to \infty} \int_{\Omega} f_n \,d\mu
        \end{align*}

\end{theorem}

\begin{theorem}{3.14}{Lebesgue Integral as Riemann Integral}

    Let $f: \mathbb{R} \to \mathbb{R}$ be a Borel-function such that:

        \begin{enumerate}[(i)]
            \setlength{\parskip}{0em}
            \item the Riemann integral $\int_{-\infty}^{\infty} f(x) \, dx$ exists and
            \item the Riemann integral $\int_{-\infty}^{\infty} |f(x)| \, dx < \infty$, i.e. is finite,
        \end{enumerate}

    then the Lebesgue integral $\int_{\mathbb{R}} f(x) \lambda(dx)$ \emph{exists} and

        \begin{align*}
            \int_{\mathbb{R}} f(x) \lambda(dx) = \int_{-\infty}^{\infty} f(x) \, dx,
        \end{align*}

    i.e. the Lebesgue integral is equal to the Riemann integral.

\end{theorem}

\begin{exercise}{3.15}{}

    Let $\nu$ be a measure that is absolutely continuous with respect to measure $\mu$ and density $g$, then $\mu(g < 0) = 0$. Moreover, $\nu$ is a probability measure $\Leftrightarrow$ $g \geq 0$ $\mu$-a.e. and $\int_{\Omega} g \, d\mu = 1$.

\end{exercise}

\begin{proposition}{3.16}{}

    Let $\nu$ and $\mu$ be measures on \SigmaAlgebra\ $\CalF$ such that $\nu$ is absolutely continuous with respect to $\mu$ and density $g$. Then for every $\CalF$-measurable function $f$ the following holds:

        \begin{align*}
            \int_{\Omega} f \, d\nu = \int_{\Omega} fg \, d\mu,
        \end{align*}

    whenever one of the integrals exists.

\end{proposition}

\begin{remark}{3.3}{}

    Let $(\Omega, \mathcal{F}, \mu)$ be measure space, $f: \Omega \to \overline{\mathbb{R}}$ \emph{non-negative} $\mathcal{F}$-measurable, then

        \begin{align*}
            \mu(f \geq \lambda) \leq \lambda^{-\alpha} \int_{\Omega} f^{\alpha} \,d\mu \quad \forall \lambda > 0, \alpha > 0.
        \end{align*}

\end{remark}

\begin{lemma}{3.10}{Fatou's Lemma}

    Let $(f_n)_{n=1}^{\infty}$ be a sequence of \emph{non-negative}, measurable functions $f: \Omega \to \overline{\mathbb{R}}$, then

        \begin{align*}
            \int_{\Omega} \liminf_{n \to \infty} f_n \, d\mu \leq \liminf_{n \to \infty} \int_{\Omega} f_n \, d\mu.
        \end{align*}

\end{lemma}

\begin{corollary}{3.11}{Fatou's Lemma Extension}

    Let $(f_n)_{n=1}^{\infty}$ be a sequence of measurable functions $f: \Omega \to \overline{\mathbb{R}}$. Then

        \begin{enumerate}[(i)]
            \setlength{\parskip}{0em}
            \item if there exists a $g \in L_1(\Omega, \CalF, \mu)$, i.e. $\int_{\Omega} |g| \, d\mu < \infty$ such that $g \leq f_n$ for all $n$, then:

            \begin{align*}
                \int_{\Omega} \liminf_{n \to \infty} f_n \, d\mu \leq \liminf_{n \to \infty} \int_{\Omega} f_n \, d\mu.
            \end{align*}
            \item if there exists a $g \in L_1(\Omega, \CalF, \mu)$, i.e. $\int_{\Omega} |g| \, d\mu < \infty$ such that $g \geq f_n$, then:

            \begin{align*}
                \int_{\Omega} \limsup_{n \to \infty} f_n \, d\mu \geq \limsup_{n \to \infty} \int_{\Omega} f_n \, d\mu.
            \end{align*}
        \end{enumerate}

\end{corollary}

\begin{theorem}{3.12}{Lebegue's Theorem on Dominated Convergence}

    Let $(f_n)_{n=1}^{\infty}$ be a sequence of Borel functions $f_n: \Omega \to \overline{\mathbb{R}}$ converging to some $f: \Omega \to \overline{\mathbb{R}}$. Assume there exists a (non-negative) Borel functions $g$ such that $|f_n| \leq g$ for any $n \geq 1$ and $\int_{\Omega} g \, d\mu < \infty$. Then the following two statements hold:

        \begin{enumerate}[(i)]
            \setlength{\parskip}{0em}
            \item
                \begin{align*}
                    \int_{\Omega} |f| \, d\mu < \infty,
                \end{align*}
            \item
                \begin{align*}
                    \int_{\Omega} f \, d\mu = \lim_{n \to \infty} \int_{\Omega} f \, d\mu.
                \end{align*}
        \end{enumerate}

\end{theorem}

\begin{proposition}{}{Restricted Expectation}

    Let $X$ be a random variable and $A \in \CalF$, then:

        \begin{align*}
            E(X \Indicator{A}) = \int_A X \,dP.
        \end{align*}

\end{proposition}

\begin{theorem}{3.17}{Integration Over The Sample Space}

    Let $f: \mathbb{R} \to \mathbb{R}$ be a Borel function and $X$ a \emph{finite} random variable, then:

        \begin{align*}
            Ef(X) = \int_{\mathbb{R}} f Q_X(dx).
        \end{align*}

\end{theorem}

\begin{proposition}{3.18}{Markov-Chebyshev's Inequality}

    Let $X$ be a \emph{non-negative} R.V., then

        \begin{align*}
            P(X \geq \lambda) \leq \lambda^{-\alpha} E(X^{\alpha}) \quad \forall \lambda > 0, \alpha > 0.
        \end{align*}

    \Hint $E(X^{\alpha}) \geq E(\Indicator{X \geq \lambda} X^{\alpha}) \geq E(\Indicator{X \geq \lambda} \lambda^{\alpha}) = \lambda^{\alpha} E(\Indicator{X \geq \lambda}) = \lambda^{\alpha} P(X \geq \lambda)$.

\end{proposition}

\begin{proposition}{3.18}{Markov-Chebyshev's Inequality (General)}

    Let $f: \Omega \to \overline{\mathbb{R}}$ be a \emph{non-negative}, measurable function, then

        \begin{align*}
            \mu(f \geq \lambda) \leq \lambda^{-\alpha} \int_{\Omega} f^{\alpha} \, d\mu \quad \forall \lambda > 0, \alpha > 0.
        \end{align*}

\end{proposition}
