%%%%%%%%%%%%%%%%%
% Basic Notions %
%%%%%%%%%%%%%%%%%

\section{Basic Notions and Notation}

\begin{example}{1.1}{}

    Simplest \SigmaAlgebra:

        \begin{itemize}
            \setlength{\parskip}{0em}
            \item $\{\emptyset, \Omega\}$, \emph{contained in every} \SigmaAlgebra \ on $\Omega$,
            \item Family of all subsets of $\Omega$, \emph{containing every} \SigmaAlgebra on $\Omega$.
        \end{itemize}

\end{example}

\begin{exercise}{1.1}{}

    Let $\CalF$ be a \SigmaAlgebra. Then $A_n \in \CalF$ for every integer $n \geq 1$ $\Rightarrow \bigcap_{n=1}^{\infty} A_n \in \CalF$.

\end{exercise}

\begin{proposition}{1.2}{}

    Let $P$ be a probability measure on \SigmaAlgebra\ $\CalF$. Then the following statements hold:

        \begin{enumerate}[(i)]
            \setlength{\parskip}{0em}
            \item $A, B \in \CalF$ s.t. $A \subseteq B$ $\Rightarrow$ $P(A) \leq P(B)$;
            \item For \emph{increasing} sequence $(A_n)_{n=1}^{\infty}$ we have

                \begin{align*}
                    \lim_{n \to \infty} P(A_n) = P\left(\bigcup_{n=1}^{\infty} A_n \right);
                \end{align*}
            \item For \emph{decreasing} sequence $(A_n)_{n=1}^{\infty}$ we have

                \begin{align*}
                    \lim_{n \to \infty} P(A_n) = P\left(\bigcap_{n=1}^{\infty} A_n \right).
                \end{align*}
        \end{enumerate}

\end{proposition}

\begin{proposition}{1.2}{General}

    Let $\mu$ be a measure on \SigmaAlgebra\ $\CalF$. Then the following statements hold:

        \begin{enumerate}[(i)]
            \setlength{\parskip}{0em}
            \item $A, B \in \CalF$ s.t. $A \subseteq B$ $\Rightarrow$ $\mu(A) \leq \mu(B)$;
            \item For \emph{increasing} sequence $(A_n)_{n=1}^{\infty}$ we have

                \begin{align*}
                    \lim_{n \to \infty} \mu(A_n) = \mu\left(\bigcup_{n=1}^{\infty} A_n \right);
                \end{align*}
            \item For \emph{decreasing} sequence $(A_n)_{n=1}^{\infty}$ we have

                \begin{align*}
                    \lim_{n \to \infty} \mu(A_n) = \mu\left(\bigcap_{n=1}^{\infty} A_n \right).
                \end{align*}
        \end{enumerate}

\end{proposition}

\begin{proposition}{}{Bounding Intersections}

    Let $A, B \in \CalF$. Then $\mu(A \cap B) \leq \mu(A)$.

    \Hint $\sigma$-additivity and $A = (A \cap B) \cup (A \setminus B)$.

\end{proposition}

\begin{proposition}{}{Measure of Set Difference, I}

    Let $A, B \in \CalF$, then $\mu(A \setminus B) = \mu(A) - \mu(A \cap B)$.

\end{proposition}

\begin{proposition}{}{Measure of Set Difference, II}

    Let $A, B \in \CalF$ and $B \subseteq A$, then $\mu(A \setminus B) = \mu(A) - \mu(B)$.

\end{proposition}

\begin{proposition}{}{Complement of Limit Inferior/Superior}

    Let $(A_n)_{n=1}^{\infty}$ be a sequence of sets in $\CalF$, then:

        \begin{enumerate}[(i)]
            \setlength{\parskip}{0em}
            \item
                \begin{align*}
                    \left(\liminf_{n \to \infty} A_n\right)^C = \limsup_{n \to \infty} A_n^C
                \end{align*}
            \item
                \begin{align*}
                    \left(\limsup_{n \to \infty} A_n\right)^C = \liminf_{n \to \infty} A_n^C
                \end{align*}
        \end{enumerate}

\end{proposition}

\begin{exercise}{Ws 2, 1}{Limit Inferior/Superior Properties}

       Let $(A_n)_{n=1}^{\infty}$ be a sequence of sets in $\CalF$, then:

        \begin{enumerate}[(i)]
            \setlength{\parskip}{0em}
            \item
                \begin{align*}
                    \liminf_{n \to \infty} A_n \coloneqq \bigcup_{n = 1}^{\infty}\bigcap_{k = n}^{\infty} A_k
                \end{align*}

                is the set of those $\omega$ that are \emph{in all but finitely many $A_n$}, i.e. that uphold the property $A_n$ captures for all except a finite amount of values of $n$.
            \item
                \begin{align*}
                    \limsup_{n \to \infty} A_n \coloneqq \bigcap_{n = 1}^{\infty}\bigcup_{k = n}^{\infty} A_k
                \end{align*}

                is the set of those $\omega$ that are \emph{in infinitely many $A_n$}, i.e. that uphold the property $A_n$ captures for an infinite amount of values of $n$.
        \end{enumerate} 

\end{exercise}

\begin{proposition}{}{Continuous Implies Borel-Measurability}

    Let $f: \mathbb{R} \to \overline{\mathbb{R}}$ be a \emph{continuous} function. Then $f$ is Borel-measurable.

\end{proposition}

\begin{proposition}{}{Countable Sets}

    Every countable subset of $\mathbb{R}$ is Borel-measurable.

\end{proposition}
