%%%%%%%%%%%%%%%%%%%%%%%%%%%
% Conditional Expectation %
%%%%%%%%%%%%%%%%%%%%%%%%%%% 

\section{Conditional Expectation}

\begin{exercise}{8.1}{}

    Let $\CalG \coloneqq \{\emptyset, \Omega\}$, i.e. the trivial \SigmaAlgebra. Then if random variable $Y$ is $\CalG$-measurable, then $Y$ is constant.

\end{exercise}

\begin{lemma}{8.2}{}

    Let $Z$ be a $\CalG$-measurable random variable such that:

        \begin{align*}
            \int_{A} Z \, dP \geq 0 \iff E(\Indicator{A}Z) \geq 0,
        \end{align*}

    for any $A \in \CalG$, then $Z \geq 0$ (a.s.).

\end{lemma}

\begin{theorem}{8.6}{Properties of Conditional Expectations}

    Let $X$ be a random variable and $\CalG \subset \CalF$ be a \SigmaAlgebra. Then the following properties hold (under the given conditions):

        \begin{enumerate}[(i)]
            \setlength{\parskip}{0em}
            \item ``Adding/Dropping Conditional Expectation'':

                \begin{align*}
                    EX = E(E(X|\CalG));
                \end{align*}
            \item ``Tower Rule'': Let $\mathcal{H} \subset \CalF$ be a \SigmaAlgebra, such that $\mathcal{H}$ \emph{contains} $\CalG$, then:

                \begin{align*}
                    E(E(X|\mathcal{H})|\CalG) = E(X|\CalG);
                \end{align*}
            \item ``Pulling/Pushing Random Variables Through'': Let $Y$ be a random variable, such that $Y$ is $\CalG$-measurable \emph{and} $E|XY| < \infty$, then:

                \begin{align*}
                    E(XY|\CalG) = YE(X|\CalG);
                \end{align*}
            \item ``Independence of Conditional'': Let $X$ and $\CalG$ be independent, i.e. $\sigma(X)$ and $\CalG$ are independent, then:

                \begin{align*}
                    E(X|\CalG) = EX.
                \end{align*}
        \end{enumerate}

\end{theorem}
