\documentclass[10pt,landscape]{article}
\usepackage[utf8]{inputenc}
\usepackage{multicol}
\usepackage{calc}
\usepackage{ifthen}
\usepackage[portrait]{geometry}
\usepackage{amsmath,amsthm,amsfonts,amssymb}
\usepackage{mathtools}
\usepackage{color,graphicx,overpic}
\usepackage{hyperref}
\usepackage{enumerate}
\usepackage{etoolbox} % Required for \appto.
\usepackage{centernot}

\usepackage{xargs}

\usepackage{ifthen}

% Define emphasis to be bold face and italic.
\DeclareTextFontCommand{\emph}{\bfseries\em}

% Removes most of whitespace above and below equations.
\newcommand{\zerodisplayskips}{%
  \setlength{\abovedisplayskip}{-5pt}% Default: 12pt plus 3pt minus 9pt
  \setlength{\belowdisplayskip}{3pt}% Default: 0pt plus 3pt
  \setlength{\abovedisplayshortskip}{-5pt}% Default: 12pt plus 3pt minus 9pt
  \setlength{\belowdisplayshortskip}{3pt}% Default: 7pt plus 3pt minus 4pt
}
\appto{\normalsize}{\zerodisplayskips}
\appto{\small}{\zerodisplayskips}
\appto{\footnotesize}{\zerodisplayskips}

%%%%%%%%%%%%%%%%%%%%%%%%%%%%%
% Theorem Environment Setup %
%%%%%%%%%%%%%%%%%%%%%%%%%%%%%
\usepackage{amsthm}

% New environments for definitions and theorems. These will let us put in the
% exact reference to the definition/theorems in the notes, e.g.
%
% \begin{definition}{5.1.1}
%     ...
% \end{definition}
%
% to create a definition with title "Definition 5.1.1", referencing the
% definition with the same number in the notes.
%\newenvironment{definition}[1] {
%    \par\addvspace{\topsep}
%    \noindent\textbf{Definition #1}.
%    \ignorespaces
%}

\newenvironmentx{observation}[2][\empty] {

    \newcommand{\Title}{Observation}

    \ifthenelse{ \equal{#2}{\empty} }{
        % Only one argument supplied, don't need parantheses.
        \par\addvspace{\topsep}
        \noindent\textbf{\Title\  #1}.
        \ignorespaces
    }{
        % Two arguments supplied, show in parantheses.
        \par\addvspace{\topsep}
        \noindent\textbf{\Title\  #1} (#2).
        \ignorespaces
    }
}


\newenvironmentx{definition}[2][\empty] {

    \newcommand{\Title}{Definition}

    \ifthenelse{ \equal{#2}{\empty} }{
        % Only one argument supplied, don't need parantheses.
        \par\addvspace{\topsep}
        \noindent\textbf{\Title\  #1}.
        \ignorespaces
    }{
        % Two arguments supplied, show in parantheses.
        \par\addvspace{\topsep}
        \noindent\textbf{\Title\  #1} (#2).
        \ignorespaces
    }
}

\newenvironmentx{theorem}[2][\empty] {

    \newcommand{\Title}{Theorem}

    \ifthenelse{ \equal{#2}{\empty} }{
        % Only one argument supplied, don't need parantheses.
        \par\addvspace{\topsep}
        \noindent\textbf{\Title\  #1}.
        \ignorespaces
    }{
        % Two arguments supplied, show in parantheses.
        \par\addvspace{\topsep}
        \noindent\textbf{\Title\  #1} (#2).
        \ignorespaces
    }
}

\newenvironmentx{lemma}[2][\empty] {

    \newcommand{\Title}{Lemma}

    \ifthenelse{ \equal{#2}{\empty} }{
        % Only one argument supplied, don't need parantheses.
        \par\addvspace{\topsep}
        \noindent\textbf{\Title\  #1}.
        \ignorespaces
    }{
        % Two arguments supplied, show in parantheses.
        \par\addvspace{\topsep}
        \noindent\textbf{\Title\  #1} (#2).
        \ignorespaces
    }
}


\newenvironmentx{proposition}[2][\empty] {

    \newcommand{\Title}{Proposition}

    \ifthenelse{ \equal{#2}{\empty} }{
        % Only one argument supplied, don't need parantheses.
        \par\addvspace{\topsep}
        \noindent\textbf{\Title\  #1}.
        \ignorespaces
    }{
        % Two arguments supplied, show in parantheses.
        \par\addvspace{\topsep}
        \noindent\textbf{\Title\  #1} (#2).
        \ignorespaces
    }
}

\newenvironmentx{corollary}[2][\empty] {

    \newcommand{\Title}{Corollary}

    \ifthenelse{ \equal{#2}{\empty} }{
        % Only one argument supplied, don't need parantheses.
        \par\addvspace{\topsep}
        \noindent\textbf{\Title\  #1}.
        \ignorespaces
    }{
        % Two arguments supplied, show in parantheses.
        \par\addvspace{\topsep}
        \noindent\textbf{\Title\  #1} (#2).
        \ignorespaces
    }
}

\newenvironmentx{remark}[2][\empty] {

    \newcommand{\Title}{Remark}

    \ifthenelse{ \equal{#2}{\empty} }{
        % Only one argument supplied, don't need parantheses.
        \par\addvspace{\topsep}
        \noindent\textbf{\Title\  #1}.
        \ignorespaces
    }{
        % Two arguments supplied, show in parantheses.
        \par\addvspace{\topsep}
        \noindent\textbf{\Title\  #1} (#2).
        \ignorespaces
    }
}

\newenvironmentx{example}[2][\empty] {

    \newcommand{\Title}{Example}

    \ifthenelse{ \equal{#2}{\empty} }{
        % Only one argument supplied, don't need parantheses.
        \par\addvspace{\topsep}
        \noindent\textbf{\Title\  #1}.
        \ignorespaces
    }{
        % Two arguments supplied, show in parantheses.
        \par\addvspace{\topsep}
        \noindent\textbf{\Title\  #1} (#2).
        \ignorespaces
    }
}

\newenvironmentx{exercise}[2][\empty] {

    \newcommand{\Title}{Exercise}

    \ifthenelse{ \equal{#2}{\empty} }{
        % Only one argument supplied, don't need parantheses.
        \par\addvspace{\topsep}
        \noindent\textbf{\Title\  #1}.
        \ignorespaces
    }{
        % Two arguments supplied, show in parantheses.
        \par\addvspace{\topsep}
        \noindent\textbf{\Title\  #1} (#2).
        \ignorespaces
    }
}

\newenvironmentx{question}[2][\empty] {

    \newcommand{\Title}{Question}

    \ifthenelse{ \equal{#2}{\empty} }{
        % Only one argument supplied, don't need parantheses.
        \par\addvspace{\topsep}
        \noindent\textbf{\Title\  #1}.
        \ignorespaces
    }{
        % Two arguments supplied, show in parantheses.
        \par\addvspace{\topsep}
        \noindent\textbf{\Title\  #1} (#2).
        \ignorespaces
    }
}

%%%%%%%%%%%%%%%%%%%%%%%%%%%%%%%%%%%%%%%%%
% Commands for Mathematical Typesetting %
%%%%%%%%%%%%%%%%%%%%%%%%%%%%%%%%%%%%%%%%%

% Redefine \leq and \geq to something nicer looking:
\renewcommand{\leq}{\leqslant}
\renewcommand{\geq}{\geqslant}

% Inner Product:
\DeclareRobustCommand{\InnerProduct}[2]{
    \ifmmode
        \left( #1,#2 \right)
    \else
        \GenericError{\space\space\space\space}
        {Attempting to use \InnerProduct outside of math mode}
    \fi
}

% Vector Norm: 
\DeclareRobustCommand{\Norm}[1]{
    \ifmmode
        \left\lVert #1 \right\rVert
    \else
        \GenericError{\space\space\space\space}
        {Attempting to use \Norm outside of math mode}
    \fi
}

% Principle Argument of Complex Number:
\DeclareMathOperator{\Arg}{Arg}

% Principle Branch of Logarithm of Complex Number:
\DeclareMathOperator{\Log}{Log}

% Image of Function:
\DeclareMathOperator{\im}{im}

% Real part of complex number:
\let\Re\relax
\DeclareMathOperator{\Re}{Re}

% Imaginary part of complex number:
\let\Im\relax
\DeclareMathOperator{\Im}{Im}

% Interior of a Curve:
\DeclareMathOperator{\Int}{Int}

% Exterior of a Curve:
\DeclareMathOperator{\Ext}{Ext}

% Residue of Function at Singularity:
\DeclareMathOperator{\Res}{Res}

% Proof Hint:
\newcommand{\Hint}{\textit{Hint: }}

% Sigma-Algebra:
\newcommand{\SigmaAlgebra}{$\sigma$-algebra}

% Indicator Function:
\newcommand{\Indicator}[1]{\mathbf{1}_{#1}}

% Calligraphic F:
\newcommand{\CalF}{\mathcal{F}}

% Floor function:
\DeclarePairedDelimiter\floor{\lfloor}{\rfloor}

% This sets page margins to .5 inch if using letter paper, and to 1cm
% if using A4 paper. (This probably isn't strictly necessary.)
% If using another size paper, use default 1cm margins.
\ifthenelse{\lengthtest { \paperwidth = 11in}}
    { \geometry{top=.5in,left=.5in,right=.5in,bottom=.5in} }
    {\ifthenelse{ \lengthtest{ \paperwidth = 297mm}}
        {\geometdry{top=1cm,left=1cm,right=1cm,bottom=1cm} }
        {\geometry{top=1cm,left=1cm,right=1cm,bottom=1cm} }
    }

% Turn off header and footer
\pagestyle{empty}

% Redefine section commands to use less space
\makeatletter
\renewcommand{\section}{\@startsection{section}{1}{0mm}%
                                {-1ex plus -.5ex minus -.2ex}%
                                {0.5ex plus .2ex}%x
                                {\normalfont\large\bfseries}}
\renewcommand{\subsection}{\@startsection{subsection}{2}{0mm}%
                                {-1explus -.5ex minus -.2ex}%
                                {0.5ex plus .2ex}%
                                {\normalfont\normalsize\bfseries}}
\renewcommand{\subsubsection}{\@startsection{subsubsection}{3}{0mm}%
                                {-1ex plus -.5ex minus -.2ex}%
                                {1ex plus .2ex}%
                                {\normalfont\small\bfseries}}
\makeatother

% Define BibTeX command
\def\BibTeX{{\rm B\kern-.05em{\sc i\kern-.025em b}\kern-.08em
    T\kern-.1667em\lower.7ex\hbox{E}\kern-.125emX}}

% Don't print section numbers
\setcounter{secnumdepth}{0}


\setlength{\parindent}{0pt}
\setlength{\parskip}{0pt plus 0.5ex}

\begin{document}
\raggedright
\footnotesize
\begin{multicols}{3}


% multicols parameters
% These lengths are set only within the two main columns
%\setlength{\columnseprule}{0.25pt}
\setlength{\premulticols}{1pt}
\setlength{\postmulticols}{1pt}
\setlength{\multicolsep}{1pt}
\setlength{\columnsep}{2pt}
\setlength{\columnseprule}{0.4pt} % For vertical lines separating columns.

\begin{center}
     \Large{Measure Theory \& Probability} \\
     \footnotesize{Sebastian Müksch, v1, 2019/20}
\end{center}

%%%%%%%%%%%%%%%%%
% Basic Notions %
%%%%%%%%%%%%%%%%%

\section{Basic Notions and Notation}

\begin{example}{1.1}{}

    Simplest \SigmaAlgebra:

        \begin{itemize}
            \setlength{\parskip}{0em}
            \item $\{\emptyset, \Omega\}$, \emph{contained in every} \SigmaAlgebra \ on $\Omega$,
            \item Family of all subsets of $\Omega$, \emph{containing every} \SigmaAlgebra on $\Omega$.
        \end{itemize}

\end{example}

\begin{exercise}{1.1}{}

    Let $\CalF$ be a \SigmaAlgebra. Then $A_n \in \CalF$ for every integer $n \geq 1$ $\Rightarrow \bigcap_{n=1}^{\infty} A_n \in \CalF$.

\end{exercise}

\begin{proposition}{1.2}{}

    Let $P$ be a probability measure on \SigmaAlgebra\ $\CalF$. Then the following statements hold:

        \begin{enumerate}[(i)]
            \item $A, B \in \CalF$ s.t. $A \subseteq B$ $\Rightarrow$ $P(A) \leq P(B)$;
            \item For \emph{increasing} sequence $(A_n)_{n=1}^{\infty}$ we have

                \begin{align*}
                    \lim_{n \to \infty} P(A_n) = P\left(\bigcup_{n=1}^{\infty} A_n \right);
                \end{align*}
            \item For \emph{decreasing} sequence $(A_n)_{n=1}^{\infty}$ we have

                \begin{align*}
                    \lim_{n \to \infty} P(A_n) = P\left(\bigcap_{n=1}^{\infty} A_n \right).
                \end{align*}
        \end{enumerate}

\end{proposition}

\begin{proposition}{1.2}{General}

    Let $\mu$ be a measure on \SigmaAlgebra\ $\CalF$. Then the following statements hold:

        \begin{enumerate}[(i)]
            \item $A, B \in \CalF$ s.t. $A \subseteq B$ $\Rightarrow$ $\mu(A) \leq \mu(B)$;
            \item For \emph{increasing} sequence $(A_n)_{n=1}^{\infty}$ we have

                \begin{align*}
                    \lim_{n \to \infty} \mu(A_n) = \mu\left(\bigcup_{n=1}^{\infty} A_n \right);
                \end{align*}
            \item For \emph{decreasing} sequence $(A_n)_{n=1}^{\infty}$ we have

                \begin{align*}
                    \lim_{n \to \infty} \mu(A_n) = \mu\left(\bigcap_{n=1}^{\infty} A_n \right).
                \end{align*}
        \end{enumerate}

\end{proposition}

\begin{proposition}{}{Bounding Intersections}

    Let $A, B \in \CalF$. Then $\mu(A \cap B) \leq \mu(A)$.

    \Hint $\sigma$-additivity and $A = (A \cap B) \cup (A \setminus B)$.

\end{proposition}

\begin{proposition}{}{Measure of Set Difference, I}

    Let $A, B \in \CalF$, then $\mu(A \setminus B) = \mu(A) - \mu(A \cap B)$.

\end{proposition}

\begin{proposition}{}{Measure of Set Difference, II}

    Let $A, B \in \CalF$ and $B \subseteq A$, then $\mu(A \setminus B) = \mu(A) - \mu(B)$.

\end{proposition}

\begin{proposition}{}{Complement of Limit Inferior/Superior}

    Let $(A_n)_{n=1}^{\infty}$ be a sequence of sets in $\CalF$, then:

        \begin{enumerate}[(i)]
            \item
                \begin{align*}
                    \left(\liminf_{n \to \infty} A_n\right)^C = \limsup_{n \to \infty} A_n^C
                \end{align*}
            \item
                \begin{align*}
                    \left(\limsup_{n \to \infty} A_n\right)^C = \liminf_{n \to \infty} A_n^C
                \end{align*}
        \end{enumerate}

\end{proposition}

\begin{exercise}{Ws 2, 1}{Limit Inferior/Superior Properties}

       Let $(A_n)_{n=1}^{\infty}$ be a sequence of sets in $\CalF$, then:

        \begin{enumerate}[(i)]
            \item
                \begin{align*}
                    \liminf_{n \to \infty} A_n \coloneqq \bigcup_{n = 1}^{\infty}\bigcap_{k = n}^{\infty} A_k
                \end{align*}

                is the set of those $\omega$ that are \emph{in all but finitely many $A_n$}, i.e. that uphold the property $A_n$ captures for all except a finite amount of values of $n$.
            \item
                \begin{align*}
                    \limsup_{n \to \infty} A_n \coloneqq \bigcap_{n = 1}^{\infty}\bigcup_{k = n}^{\infty} A_k
                \end{align*}

                is the set of those $\omega$ that are \emph{in infinitely many $A_n$}, i.e. that uphold the property $A_n$ captures for an infinite amount of values of $n$.
        \end{enumerate} 

\end{exercise}

%%%%%%%%%%%%%%%%%%%%%%%%
% Expectation Integral %
%%%%%%%%%%%%%%%%%%%%%%%%

\section{Expectation Integrals}

\begin{proposition}{(Unknown)}{}

    Let $A,B \subseteq \Omega$. Then the following equalities hold:

        \begin{itemize}
            \setlength{\parskip}{0em}
            \item $\Indicator{A^C} = 1 - \Indicator{A}$,
            \item $\Indicator{A \cap B} = \Indicator{A}\Indicator{B}$.
            \item $\Indicator{A \cup B} = \Indicator{A} + \Indicator{B} - \Indicator{A \cap B}$.
        \end{itemize}

\end{proposition}

\begin{lemma}{3.3}{}

    Let $X$ be a \emph{non-negative} random variable. Then there exists a sequence of \emph{non-negative, simple} random variables $X_n$ converging to $X$ for every $\omega \in \Omega$.

    \Hint $h_n(x)=\min\{\floor{2^nx}/2^n, n\}$ is non-negative, simple and increasing, approaching $x$. Consider $X_n \coloneqq h(X) \to X$.

\end{lemma}

\begin{lemma}{}{Simple Function Integral Properties}

    Let $f,g: \Omega \to \overline{\mathbb{R}}$ be a \emph{non-negative}, simple functions and $a,b \geq 0$. Then the following holds:

    \begin{itemize}
        \item $\int_{\Omega} f \, d\mu \geq 0$,
        \item $\int_{\Omega} (af + bg) \, d\mu = a\int_{\Omega} f + b\int_{\Omega}g \, d\mu$.
    \end{itemize}

\end{lemma}

\begin{corollary}{}{Positive Integral over Set}

    Let $A \subseteq \Omega$ and $f: \Omega \to \overline{\mathbb{R}}$ a \emph{non-negative} measurable function. Then $\int_A f \, d\mu \geq 0$.

\end{corollary}

\begin{lemma}{3.3}{General}

    Let $f: \Omega \to \overline{\mathbb{R}}$ be a \emph{non-negative}, measurable function. The there exists a sequence $f_n$ of \emph{non-negative}, simple functions such that:

    \begin{align*}
        \lim_{n \to \infty} f_n = f
    \end{align*}

    \Hint Use $h_n$ from Lemma 3.3's hint.

\end{lemma}

\begin{exercise}{3.5}{}

    Let $A \in \CalF$ s.t. $\mu(A) = 0$. Then for \emph{any} measurable function $f: \Omega \to \overline{\mathbb{R}}$:
    
        \begin{align*}
            \int_A f \,d\mu = 0.
        \end{align*}

\end{exercise}

\begin{exercise}{3.6}{}

    Let $f: \Omega \to \mathbb{R}$ be a measurable function, then:

        \begin{enumerate}[(i)]
            \setlength{\parskip}{0em}
            \item For any $c \in \mathbb{R}$ and $A \in \CalF$:

                \begin{align*}
                    \int_A cf \, d\mu = c \int_A f \, d\mu,
                \end{align*}
            
            provided the integral exists.
            \item For any $A, B \in \CalF$, such that $A \cap B = \emptyset$:

                \begin{align*}
                    \int_{A \cup B} f \,d\mu = \int_{A} f \,d\mu + \int_{B} f \,d\mu,
                \end{align*}

            provided the left-hand or right-hand side is well-defined.
        \end{enumerate}

\end{exercise}

\begin{theorem}{3.8}{Monotone Convergence}

    Let $(f_n)_{n=1}^{\infty}$ be increasing sequence of non-negative, measurable functions $f_n: \Omega \to \overline{\mathbb{R}}$, converging to some $f$. Then:

        \begin{align*}
            \int_{\Omega} \lim_{n \to \infty} f_n \,d\mu = \lim_{n \to \infty} \int_{\Omega} f_n \,d\mu
        \end{align*}

\end{theorem}

\begin{exercise}{3.15}{}

    Let $\nu$ be a measure that is absolutely continuous with respect to measure $\mu$ and density $g$, then $\mu(g < 0) = 0$. Moreover, $\nu$ is a probability measure $\Leftrightarrow$ $g \geq 0$ $\mu$-a.e. and $\int_{\Omega} g \, d\mu = 1$.

\end{exercise}

\begin{proposition}{3.16}{}

    Let $\nu$ and $\mu$ be measures on \SigmaAlgebra\ $\CalF$ such that $\nu$ is absolutely continuous with respect to $\mu$ and density $g$. Then for every $\CalF$-measurable function $f$ the following holds:

        \begin{align*}
            \int_{\Omega} f \, d\nu = \int_{\Omega} fg \, d\mu,
        \end{align*}

    whenever one of the integrals exists.

\end{proposition}

\begin{proposition}{3.18}{Markov-Chebyshev's Inequality}

    Let $X$ be a \emph{non-negative} R.V., then

        \begin{align*}
            P(X \geq \lambda) \leq \lambda^{-\alpha} E(X^{\alpha}) \quad \forall \lambda > 0, \alpha > 0.
        \end{align*}

\end{proposition}

\begin{remark}{3.3}{}

    Let $(\Omega, \mathcal{F}, \mu)$ be measure space, $f: \Omega \to \overline{\mathbb{R}}$ \emph{non-negative} $\mathcal{F}$-measurable, then

        \begin{align*}
            \mu(f \geq \lambda) \leq \lambda^{-\alpha} \int_{\Omega} f^{\alpha} \,d\mu \quad \forall \lambda > 0, \alpha > 0.
        \end{align*}

\end{remark}

\begin{lemma}{3.10}{Fatou's Lemma}

    Let $(f_n)_{n=1}^{\infty}$ be a sequence of \emph{non-negative}, measurable functions $f: \Omega \to \overline{\mathbb{R}}$, then

        \begin{align*}
            \int_{\Omega} \liminf_{n \to \infty} f_n \, d\mu \leq \liminf_{n \to \infty} \int_{\Omega} f_n \, d\mu.
        \end{align*}

\end{lemma}

\begin{corollary}{3.11}{Fatou's Lemma Extension}

    Let $(f_n)_{n=1}^{\infty}$ be a sequence of measurable functions $f: \Omega \to \overline{\mathbb{R}}$. Then

        \begin{enumerate}[(i)]
            \item if there exists a $g \in L_1(\Omega, \CalF, \mu)$, i.e. $\int_{\Omega} |g| \, d\mu < \infty$ such that $g \leq f_n$ for all $n$, then:

            \begin{align*}
                \int_{\Omega} \liminf_{n \to \infty} f_n \, d\mu \leq \liminf_{n \to \infty} \int_{\Omega} f_n \, d\mu.
            \end{align*}
            \item if there exists a $g \in L_1(\Omega, \CalF, \mu)$, i.e. $\int_{\Omega} |g| \, d\mu < \infty$ such that $g \geq f_n$, then:

            \begin{align*}
                \int_{\Omega} \limsup_{n \to \infty} f_n \, d\mu \geq \limsup_{n \to \infty} \int_{\Omega} f_n \, d\mu.
            \end{align*}
        \end{enumerate}

    \Hint TODO

\end{corollary}

\begin{theorem}{3.12}{Lebegue's Theorem on Dominated Convergence}

    Let $(f_n)_{n=1}^{\infty}$ be a sequence of Borel functions $f_n: \Omega \to \overline{\mathbb{R}}$ converging to some $f: \Omega \to \overline{\mathbb{R}}$. Assume there exists a (non-negative) Borel functions $g$ such that $|f_n| \leq g$ for any $n \geq 1$ and $\int_{\Omega} g \, d\mu < \infty$. Then the following two statements hold:

        \begin{enumerate}[(i)]
            \item
                \begin{align*}
                    \int_{\Omega} |f| \, d\mu < \infty,
                \end{align*}
            \item
                \begin{align*}
                    \int_{\Omega} f \, d\mu = \lim_{n \to \infty} \int_{\Omega} f \, d\mu.
                \end{align*}
        \end{enumerate}

    \Hint TODO

\end{theorem}

\begin{proposition}{}{Restricted Expectation}

    Let $X$ be a random variable and $A \in \CalF$, then:

        \begin{align*}
            E(X \Indicator{A}) = \int_A X \,dP.
        \end{align*}

\end{proposition}

\begin{lemma}{4.4}{Borel-Cantelli Lemma}

    Let $(A)_{n=1}^{\infty}$ be a sequence of sets $A_n \in \CalF$ such that $\sum_{n=1}^{\infty} \mu(A_n) < \infty$, i.e. the series of measures of $A_n$ converges. Then for:

        \begin{align*}
            A \coloneqq \limsup_{n \to \infty} A_n \coloneqq \bigcap_{n=1}^{\infty} \bigcup_{k=n}^{\infty} A_k,
        \end{align*}

    we have $\mu(A) = 0$.

    \Hint Define $B_n \coloneqq \bigcup_{k=n}^{\infty} A_k$, then $(B_n)_{n=1}^\infty$ is decreasing and so $\bigcap_{n=1}^{\infty} B_n = \lim_{n \to \infty} B_n$ and realize that $\sum_{n=1}^{\infty} \mu(A_n) < \infty$ $\Rightarrow$ tail sums $\sum_{k=n}^{\infty} \mu(A_k) \to 0$ as $n \to \infty$.

\end{lemma}

%%%%%%%%%%%%%%%%%%%%%%%%%%%%%%%%%%%%%%%
% Convergence of Measurable Functions %
%%%%%%%%%%%%%%%%%%%%%%%%%%%%%%%%%%%%%%%

\section{Convergence of Measurable Functions}

\begin{exercise}{5.2}{Almost Finite, Converging Sequence is Bounded}

    Assume that $\mu(\Omega) < \infty$. Let $(f_n)_{n=1}^{\infty}$ be \emph{$\mu$-a.e. finite}, converging in measure to $\mu$ to some $f: \Omega \to \mathbb{R}$. Then the sequence of $f_n$ is \emph{bounded in measure $\mu$, uniformly in $n$}, i.e.:

        \begin{align*}
            \lim_{K \to \infty} \sup_{n \geq 1} \mu(|f_n| \geq K) = 0.
        \end{align*}

    \Hint $f_n$ \emph{$\mu$-a.e. finite} \emph{and} $\mu(\Omega) < \infty$ $\Rightarrow$ $f_n$ bounded in measure (not necessarily uniformly), so

        \begin{align*}
            \lim_{K \to \infty} \sup_{n \geq 1} \mu(|f_n| \geq K) = \\ \lim_{K \to \infty} \limsup_{n \to \infty} \mu(|f_n| \geq K).
        \end{align*}

    Then use observation of splitting measures of inequalities.

\end{exercise}

\begin{exercise}{5.3}{Product of Bounded \& Zero Convergent is Zero Convergent}

    Let $(f_n)_{n=1}^{\infty}$ and $(g_n)_{n=1}^{\infty}$ be sequences of $\mu$-a.e. finite measurable functions such that the $f_n$ are bounded in measure $\mu$, uniformly in $n$ and $g_n \to 0$ in measure $\mu$, as $n \to \infty$. Then $f_ng_n \to 0$ in measure $\mu$, as $n \to \infty$.

\end{exercise}

\begin{exercise}{Ws 3, 1}{}

    Let $\mu-\lim f_n = f$, then there exists a subsequence $(f_{n_k})_{k=1}^{\infty}$ such that $(n_k)_{k=1}^{\infty}$ is increasing and $f_{n_k} \to f$ ($\mu$-a.e.).

    \Hint Borel-Cantelli with $A_k = \{ |f_{n_k} - f| \geq 1/k \}$ s.t. $\mu(A_k) \leq 1/k^2$.

\end{exercise}

\begin{theorem}{5.4}{Measure Convergence Has Almost Everywhere Converging Subsequence}

    Let $(f_n)_{n=1}^{\infty}$ be a sequence of functions converging in measure $\mu$ to some $\mu$-a.e. finite function $f$. Then there exists a (strictly) increasing sequence $(n_k)_{k=1}^{\infty}$ of positive integers such that $\lim_{k \to \infty} f_{n_k} = f$ $\mu$-almost everywhere.

\end{theorem}

\begin{exercise}{5.5}{}

    Convergence in measure $\mu$ does not imple convergence $\mu$-almost everywhere.

    \Hint $(\mathbb{R}, \mathcal{B}(\mathbb{R}), \lambda)$ with $f_n = \Indicator{[k/2^m, (k+1)/2^m]}$ where $k = 0,1,\hdots,2^m - 1$ and $m=0,1,\hdots$ such that $n = 2^m + k$.

\end{exercise}

\begin{exercise}{Ws 3, 2}{Convergence Implication}

    Let $\mu(\Omega) < \infty$. Then $\lim_{n \to \infty} f_n = f$ ($\mu$-a.e.) $\Rightarrow$ $\mu-\lim_{n \to \infty} f_n = f$.

\end{exercise}

\begin{exercise}{Ws 3, 3}{Relaxed Domnitated Convergence}

    Lebegue's Theorem on Dominated convergence holds under the following, relaxed conditions:

        \begin{enumerate}[(i)]
            \item $\lim_{n \to \infty} f_n = f$ \emph{$\mu$-a.e.},  $|f_n| \leq g|$ $\mu$-a.e. and $g \in L_1(\Omega, \CalF, \mu)$, i.e. $\int_{\Omega} |g| \, d\mu < \infty$; and
            \item $\mu-\lim_{n \to \infty} f_n = f$,  $|f_n| \leq g|$ $\mu$-a.e. and $g \in L_1(\Omega, \CalF, \mu)$, i.e. $\int_{\Omega} |g| \, d\mu < \infty$.
        \end{enumerate}

    \Hint TODO

\end{exercise}

%%%%%%%%%%%%%%%%%%%%%%%%%%%%%%%%%%%%%%%%%%%%%%%%%%%%%%%%%%%%%%%
% Independence of events and independence of random variables %
%%%%%%%%%%%%%%%%%%%%%%%%%%%%%%%%%%%%%%%%%%%%%%%%%%%%%%%%%%%%%%%

\section{Independence of Events and Random Variables}

\begin{theorem}{6.7}{Fubini-Tonelli Theorem}

    Let $(\Omega_i, \CalF_i, \mu_i)$, for $i = 1,2$, be measure spaces and $(\Omega, \CalF, \mu)$ be the product measure space of the two, i.e. $\Omega = \Omega_1 \times \Omega_2$, $\CalF = \CalF_i \otimes \CalF_2$ and $\mu = \mu_1 \otimes \mu_2$. Let $f: \Omega \to \overline{\mathbb{R}}$ be a \emph{non-negative} $\CalF$-measurable function. If $\mu_i$, for $i = 1,2$, are \emph{finite measures} on $\Omega_i$, for $i = 1,2$, respectively, then the following iterated integrals are well-defined and:

        \begin{align*}
            \int_{\Omega_1 \times \Omega_2} f \, d\mu_1 \otimes \mu_2 &= \int_{\Omega_1} \int_{\Omega_2} f \, d\mu_2 d\mu_1 = \\ &= \int_{\Omega_2} \int_{\Omega_1} f \, d\mu_1 d\mu_2.
        \end{align*}

    Furthermore, this statement holds for $\CalF$-measurable functions if:

        \begin{align*}
            \int_{\Omega_1 \times \Omega_2} |f| \, d\mu_1 \otimes \mu_2 < \infty.
        \end{align*}

\end{theorem}

%%%%%%%%%%%%%%%%%%%%%%%%%%%
% Conditional Expectation %
%%%%%%%%%%%%%%%%%%%%%%%%%%%

%%%%%%%%%%%%%%%
% Definitions %
%%%%%%%%%%%%%%%

\section{Definitions}

\subsection{Basic Notions and Notation}

\textit{In the following, $\Omega$ is a set, $\CalF$ a \SigmaAlgebra\ on $\Omega$. If used, then $\mu$ is a measure. Otherwise, the measure is the probability measure $P$.}

\begin{definition}{1.1}{}

    Let $\mathcal{F}$ be a family of subsets of set $\Omega$. $\mathcal{F}$ is called a \emph{$\sigma$-algebra} if:

        \begin{itemize}
            \setlength{\parskip}{0em}
            \item \emph{Closed Under Complement}: $A \in \mathcal{F} \Rightarrow A^c \in \mathcal{F}$,
            \item \emph{Closed Under Arbitrary Union}: $A_n \in \mathcal{F}$ for integer $n \geq 1$ $\Rightarrow \bigcup_{n=1}^{\infty}A_n \in \mathcal{F}$,
            \item \emph{Contains Entire Set}: $\Omega \in \mathcal{F}$
        \end{itemize}

\end{definition}

\begin{definition}{1.2}{}
    Let $\mathcal{C}$ be a family of subsets of $\Omega$. There exists a $\sigma$-algebra which contains $\mathcal{C}$ \emph{and} which is contained in every $\sigma$-algebra that contains $\mathcal{C}$ (take intersection of all $\sigma$-algebras. Such $\sigma$-algebra is \emph{unique} and called \emph{smallest $\sigma$-algebra containing $\mathcal{C}$} or \emph{$\sigma$-algebra generated by $\mathcal{C}$}, denoted by $\sigma(\mathcal{C})$. Simplest example, let $A \subseteq \Omega$:

        \begin{align*}
            \sigma(A) = \{\emptyset, A, A^c, \Omega\}.
        \end{align*}
\end{definition}

\begin{definition}{}{Finite Measure Space}

    Let $(\Omega, \CalF, \mu)$ be a measure space. If $\mu(\Omega) < \infty$, then we call the measure space \emph{finite}.

\end{definition}

\subsection{Random Variables}

\begin{definition}{2.1.1}{}

    Let $A \subseteq \Omega$ and $\Indicator{A}$ be defined as follows:

        \begin{align*}
            \Indicator{A}(\omega) =
            \begin{cases}
                1, & \omega \in A \\
                0, & \omega \not\in A
            \end{cases}.
        \end{align*}

    Then $\Indicator{A}$ is a R.V. and called the \emph{indicator (function) of (events) A}.

\end{definition}

\subsection{Expextation Integrals}

\begin{definition}{}{Indicator Integral}

    Let $A \subseteq \Omega$, then:

        \begin{align*}
            \int_{\Omega} \Indicator{A} \,d\mu = \mu(A).
        \end{align*}

\end{definition}

\begin{definition}{}{Simple Function}

    Let $f: \Omega \to \mathbb{R}$ be a \emph{simple function}, then $f$ takes finitely many values. Formally, if $I$ is a finite index set, $(A_i)_{i \in I}$ a famility of \emph{disjoint} subsets of $\Omega$ and $(c_i)_{i \in I}$ a family of real numbers, then:

        \begin{align*}
            f(\omega) = \sum_{i \in I} c_i \Indicator{A_i}(\omega).
        \end{align*}

\end{definition}

\begin{definition}{}{Lebesgue Integral for Expectation}

    Let $X$ be a random variable. Then we write:

        \begin{align*}
            EX = \int_{\Omega} X \,dP.
        \end{align*}

\end{definition}

\begin{definition}{}{Non-negative, Measurable Lebesgue Integral}

    Let $f: \Omega \to \overline{\mathbb{R}}$ be a \emph{non-negative}, measurable function and $(f_n)_{n=1}^{\infty}$ a sequence of \emph{non-negative, simple} functions sucht that $\lim_{n \to \infty} f_n = f$. Then

        \begin{align*}
            \int_{\Omega} f \, d\mu = \lim_{n \to \infty} f_n \, d\mu.
        \end{align*}

\end{definition}

\begin{definition}{}{Lebesgue Integral}

     Let $f: \Omega \to \overline{\mathbb{R}}$ be a measurable function. The \emph{Lebesgue Integral} of $f$ is defined as:

        \begin{align*}
            \int_{\Omega} f \,d\mu = \int_{\Omega} f^+ \,d\mu - \int_{\Omega} f^- \,d\mu,
        \end{align*}

    where $f^+ = \max\{f, 0\}$ and $f^-=\max\{-f, 0\}$, if at least one of the integrals on the right-hand side is finite. If both are infinite, then we say that the Lebesgue Integral of $f$ does not exist.

\end{definition}

\begin{definition}{}{Restricted Integration}

    Let $A \in \CalF$ and $f: \Omega \to \overline{\mathbb{R}}$ is a measurable function, then we define:

        \begin{align*}
            \int_A f \, d\mu = \int_{\Omega} \Indicator{A} f \, d\mu,
        \end{align*}

    when the integral of $\Indicator{A} f$ w.r.t $\mu$ exists.

\end{definition}

\begin{definition}{3.7}{Absolute Continuity}

    Let $\mu$ and $\nu$ be measures on \SigmaAlgebra\ $\CalF$ such that for some $\CalF$-measureable $g: \Omega \to \mathbb{R}$:

        \begin{align*}
            \nu(A) = \int_{\Omega} \Indicator{A} g \, d\mu = \int_{A} g \mu(dx),
        \end{align*}

    for all $A \in \CalF$. Then $\nu$ is called \emph{absolutely continuous} with respect to $\mu$ and $g$ is called the \emph{density} or \emph{Radon-Nikodym derivative} (Notation: $g = \frac{d\nu}{d\mu}$).

\end{definition}

\subsection{Convergence of Measurable Functions}

\begin{definition}{}{$\mu$-Almost Everywhere Finite}

    Let $f: \Omega \to \overline{\mathbb{R}}$ be $\CalF$-measurable, then $f$ is said to be \emph{$\mu$-almost everywhere} ($\mu$-a.e.) finite if $\mu(|f| = \infty) = 0$.

\end{definition}

\begin{definition}{}{Almost Surely Finite}

    Let $f: \Omega \to \overline{\mathbb{R}}$ be $\CalF$-measurable, then $f$ is said to be \emph{almost surely} (a.s.) finite if $P(|f| = \infty) = 0$ $\Leftrightarrow P(|f| < \infty) = 1$.

\end{definition}

\begin{definition}{5.1}{$\mu$-Almost Everywhere Convergence}

    Let $(f_n)_{n=1}^{\infty}$ be $\CalF$-measurable functions. The $f_n$ are said to \emph{converge $\mu$-almost everywhere} to a \emph{$\mu$-a.e. finite} $f: \Omega \to \overline{\mathbb{R}}$ as $n \to \infty$ if there exists an $A \in \CalF$ s.t. $\mu(A) = 0$ and

        \begin{align*}
            \lim_{n \to \infty} f_n(\omega) = f(\omega) \in \mathbb{R}, \quad \forall \omega \in A^C.
        \end{align*}

    \emph{Notation:} $\lim_{n \to \infty} f_n = f$ ($\mu$-a.e.) or $f_n \to f$ ($\mu$-a.e.).

\end{definition}

\begin{definition}{5.1}{Almost Sure Convergence}

    Let $(f_n)_{n=1}^{\infty}$ be $\CalF$-measurable functions. The $f_n$ are said to \emph{converge almost surely} to a \emph{a.s. finite} $f: \Omega \to \overline{\mathbb{R}}$ as $n \to \infty$ if there exists an $A \in \CalF$ s.t. $P(A) = 0$ and

        \begin{align*}
            \lim_{n \to \infty} f_n(\omega) = f(\omega) \in \mathbb{R}, \quad \forall \omega \in A^C.
        \end{align*}

    \emph{Notation:} $\lim_{n \to \infty} f_n = f$ (a.s.) or $f_n \to f$ (a.s.).

\end{definition}

\begin{definition}{5.2}{Convergence in Measure}

    Let $(f_n)_{n=1}^{\infty}$ be $\CalF$-measurable functions. The $f_n$ are said to \emph{converge in measure $\mu$} to a \emph{$\mu$-a.e. finite} $f: \Omega \to \overline{\mathbb{R}}$ as $n \to \infty$ if

        \begin{align*}
            \lim_{n \to \infty} \mu(|f_n - f| \geq \varepsilon) = 0, \quad \forall \varepsilon > 0.
        \end{align*}

    \emph{Notation:} $\mu-\lim_{n \to \infty} f_n = f$.

\end{definition}

\begin{definition}{5.2}{Convergence in Probability}

    Let $(f_n)_{n=1}^{\infty}$ be $\CalF$-measurable functions. The $f_n$ are said to \emph{converge in probability} to a \emph{a.s. finite} $f: \Omega \to \overline{\mathbb{R}}$ as $n \to \infty$ if

        \begin{align*}
            \lim_{n \to \infty} P(|f_n - f| \geq \varepsilon) = 0, \quad \forall \varepsilon > 0.
        \end{align*}

\end{definition}

\begin{definition}{}{Bounded in Measure}

    Let $(f_n)_{n=1}^{\infty}$ be a sequence of measurable functions, then it is \emph{bounded in measure $\mu$} if

        \begin{align*}
            \lim_{K \to \infty} \mu(|f_n| \geq K) = 0,
        \end{align*}

    for any $n \geq 1$.

\end{definition}

\begin{definition}{}{Bounded Uniformly in Measure}

    Let $(f_n)_{n=1}^{\infty}$ be a sequence of measurable functions, then it is \emph{bounded in measure $\mu$, uniformly in $n$} if

        \begin{align*}
            \lim_{K \to \infty} \sup_{n \geq 1} \mu(|f_n| \geq K) = 0.
        \end{align*}

\end{definition}

\begin{definition}{}{Finite Second Moment}

    Let $X$ be a random variable. Then $X$ has \emph{finite second moment} if $EX^2 < \infty$.

\end{definition}

\subsection{Conditional Expectation}

\begin{definition}{}{Sub-$\sigma$-Algebra Measurable}

    Let $Y$ be a random variable TODO FINISH

\end{definition}

%%%%%%%%%%%%%%%%%%%%%%
% Useful Observation %
%%%%%%%%%%%%%%%%%%%%%%

\section{Useful Observations}

\begin{observation}{}{Bounding Measures}

    The following inequalities to bound measures are \emph{always} applicable, for \emph{any} sets $A, B, C \in \CalF$:

    \begin{enumerate}
        \item ``Dropping a set in an intersection gives an upper bound'' $\Leftrightarrow$ ``Relaxing constraints'':

            \begin{align*}
                \mu(A \cap B) \leq \mu(A).
            \end{align*}
        \item ``Dropping a set in a union gives an lower bound'':

            \begin{align*}
                \mu(A \cup B) \geq \mu(A).
            \end{align*}
        \item ``Adding a set in a union gives an upper bound'' $\Leftrightarrow$ ``Adding constraints'':

            \begin{align*}
                \mu(A \cup B) \leq \mu(A \cup B \cup C).
            \end{align*}
        \item ``Intersections are less than a set and a set is less than a union'':

            \begin{align*}
                \mu(A \cap B) \leq \mu(A) \leq \mu(A \cup B).
            \end{align*}
    \end{enumerate}

\end{observation}

\begin{observation}{}{Adding $\Omega$ by Intersection}

    If you would like to introduce a property to an existing set $A$ to make it easier to work with, for instance easier to bound, you can add an intersection with $\Omega$:

        \begin{align*}
            \mu(A) = \mu(\Omega \cap A).
        \end{align*}

    Then $\Omega$ can be split into the set $B$ that represents the property and $B^C$ that does not have the property, where $\Omega = B \cup B^C$. Then:

        \begin{align*}
            \mu(A) = \mu(\Omega \cap A) = \mu((B \cup B^C) \cap A) = \\
            \mu((B \cup B^C) \cap A) = \mu((B \cap A) \cup (B^C \cap A)).
        \end{align*}

    Using $\sigma$-additivity, we get:

        \begin{align*}
            \mu(A) = \mu(B \cap A) + \mu(B^C \cap A).
        \end{align*}

    Then by the observation on bounding measures, this can be made into an inequality:

        \begin{align*}
            \mu(A) &= \mu(B \cap A) + \mu(B^C \cap A) \\
            &\leq \mu(B \cap A) + \mu(B^C).
        \end{align*}

\end{observation}

\begin{observation}{}{Increasing Sequence of Sets}

    For an \emph{increasing} sequence of sets $(A_n)_{n=1}^{\infty}$ we can define:

        \begin{align*}
            \lim_{n \to \infty} A_n \coloneqq \bigcup_{n=1}^{\infty} A_n
        \end{align*}

\end{observation}

\begin{observation}{}{Decreasing Sequence of Sets}

    For an \emph{decreasing} sequence of sets $(A_n)_{n=1}^{\infty}$ we can define:

        \begin{align*}
            \lim_{n \to \infty} A_n \coloneqq \bigcap_{n=1}^{\infty} A_n
        \end{align*}

\end{observation}

\begin{observation}{}{$\mu$-Almost Everywhere Finite, I}

    If $f: \Omega \to \mathbb{R}$ is $\mu$-a. e. finite, then note that if $A_n \coloneqq \{ |f| \geq n \}$, then $(A_n)_{n=1}^{\infty}$ is a decreasing sequence and so:

        \begin{align*}
            \mu\left(\bigcap_{n=1}^{\infty} A_n\right) = \mu\left(\lim_{n \to \infty} A_n\right) = \mu(|f| = \infty) \\ = 0.
        \end{align*}

\end{observation}

\begin{observation}{}{$\mu$-Almost Everywhere Finite, II}

    If $f: \Omega \to \mathbb{R}$ is $\mu$-a. e. finite, then observe

        \begin{align*}
            \mu(|f| = \infty) = \lim_{R \to \infty} \mu(|f| \geq R) = 0.
        \end{align*}

\end{observation}

\begin{observation}{}{Almost Surely Finite, II}

    If $f: \Omega \to \mathbb{R}$ is a.s. finite, then observe

        \begin{align*}
            P(|f| = \infty) = \lim_{R \to \infty} P(|f| \geq R) = 0. \\
            \iff P(|f| < \infty) = \lim_{R \to \infty} P(|f| < R) = 1.
        \end{align*}

\end{observation}

\begin{observation}{}{Almost Surely Finite}

    If $f: \Omega \to \mathbb{R}$ is a. s. finite, then note that if $A_n \coloneqq \{ |f| \geq n \}$, then $(A_n)_{n=1}^{\infty}$ is a decreasing sequence and so:

        \begin{align*}
            P\left(\bigcap_{n=1}^{\infty} A_n\right) = P\left(\lim_{n \to \infty} A_n\right) = P(|f| = \infty) \\ = 0.
        \end{align*}

\end{observation}

\begin{observation}{}{$\mu$-Almost Everywhere Convergence I}

    If $f_n \to f$ $\mu$-a.e., then $\mu(f_n \not\to f) = 0$.

\end{observation}

\begin{observation}{}{$\mu$-Almost Everywhere Convergence II}

    If $A \in \CalF$ is a set such that $\mu(A) = 0$ and

        \begin{align*}
            \lim_{n \to \infty} |f_n(\omega) - f(\omega)| = 0 \quad \forall \omega \in A^C,
        \end{align*}

    then $f_n \to f$ $\mu$-almost everywhere.

\end{observation}

\begin{observation}{}{Almost Sure Convergence}

    If $f_n \to f$ a.s., then $P(f_n \not\to f) = 0$ or equivalently $P(f_n \to f) = 1$.

\end{observation}

\begin{observation}{}{Splitting Measures of Inequalities}

    Let $f,g$ be measurable functions and $a \in \mathbb{R}$, then observe that:

        \begin{align*}
            \mu(|f| \geq a) \leq \mu\left( |f - g| \geq \frac{a}{2} \right) + \mu\left( |g| \geq \frac{a}{2} \right)
        \end{align*}

\end{observation}

\begin{observation}{}{Using Borel-Cantelli}

    If you can define sets $(A_k)_{k=1}^{\infty}$ such that $\mu(A_k) \leq 1/k^2$, then you can use Borel-Cantelli as:

        \begin{align*}
            \sum_{k=1}^{\infty} \mu(A_k) \leq \sum_{k=1}^{\infty} \frac{1}{k^2} < \infty.
        \end{align*}

    In fact, the choice of $1/k^2$ is more or less arbitrary. This technique would work with any $r_k$ s.t. $\sum_{k=1}^{\infty} r_k < \infty$ and $\mu(A_k) \leq r_k$. Caution: $r_k = 1/k$ does \emph{not} work.

\end{observation}

\begin{observation}{}{Function As Integral}

    Let $f: \Omega \to \overline{\mathbb{R}}$ be a \emph{non-negative} measurable function, the obvserve that

        \begin{align*}
            f(\omega) = \int\limits_0^{f(\omega)} \, dx = \int\limits_0^{\infty} \Indicator{x \leq f(\omega)} \, dx
        \end{align*}

\end{observation}

\begin{observation}{}{Bounding Complement Probabilities}

    Note that $1 - x \leq e^{-x}$. Therefore, we can bound probabilities of a product of complement events, for instance:

        \begin{align*}
            \prod_{n = 1}^{\infty} P(A_n^C) = \prod_{n = 1}^{\infty} [1 - P(A_n)] \leq \\ \prod_{n = 1}^{\infty} e^{-P(A_n)} = e^{\sum_{n = 1}^{\infty} -P(A_n)}
        \end{align*}

\end{observation}

\end{multicols}

\end{document}
