\documentclass[10pt,landscape]{article}
\usepackage[utf8]{inputenc}
\usepackage{multicol}
\usepackage{calc}
\usepackage{ifthen}
\usepackage[portrait]{geometry}
\usepackage{amsmath,amsthm,amsfonts,amssymb}
\usepackage{mathtools}
\usepackage{color,graphicx,overpic}
\usepackage{hyperref}
\usepackage{enumerate}
\usepackage{etoolbox} % Required for \appto.
\usepackage{centernot}

\usepackage{xargs}

\usepackage{ifthen}

% Define emphasis to be bold face and italic.
\DeclareTextFontCommand{\emph}{\bfseries\em}

% Removes most of whitespace above and below equations.
\newcommand{\zerodisplayskips}{%
  \setlength{\abovedisplayskip}{-5pt}% Default: 12pt plus 3pt minus 9pt
  \setlength{\belowdisplayskip}{3pt}% Default: 0pt plus 3pt
  \setlength{\abovedisplayshortskip}{-5pt}% Default: 12pt plus 3pt minus 9pt
  \setlength{\belowdisplayshortskip}{3pt}% Default: 7pt plus 3pt minus 4pt
}
\appto{\normalsize}{\zerodisplayskips}
\appto{\small}{\zerodisplayskips}
\appto{\footnotesize}{\zerodisplayskips}

%%%%%%%%%%%%%%%%%%%%%%%%%%%%%
% Theorem Environment Setup %
%%%%%%%%%%%%%%%%%%%%%%%%%%%%%
\usepackage{amsthm}

% New environments for definitions and theorems. These will let us put in the
% exact reference to the definition/theorems in the notes, e.g.
%
% \begin{definition}{5.1.1}
%     ...
% \end{definition}
%
% to create a definition with title "Definition 5.1.1", referencing the
% definition with the same number in the notes.
\newenvironment{definition}[1] {
    \par\addvspace{\topsep}
    \noindent\textbf{Definition #1}.
    \ignorespaces
}

\newenvironmentx{theorem}[2][\empty] {

    \newcommand{\Title}{Theorem}

    \ifthenelse{ \equal{#2}{\empty} }{
        % Only one argument supplied, don't need parantheses.
        \par\addvspace{\topsep}
        \noindent\textbf{\Title\  #1}.
        \ignorespaces
    }{
        % Two arguments supplied, show in parantheses.
        \par\addvspace{\topsep}
        \noindent\textbf{\Title\  #1} (#2).
        \ignorespaces
    }
}

\newenvironmentx{lemma}[2][\empty] {

    \newcommand{\Title}{Lemma}

    \ifthenelse{ \equal{#2}{\empty} }{
        % Only one argument supplied, don't need parantheses.
        \par\addvspace{\topsep}
        \noindent\textbf{\Title\  #1}.
        \ignorespaces
    }{
        % Two arguments supplied, show in parantheses.
        \par\addvspace{\topsep}
        \noindent\textbf{\Title\  #1} (#2).
        \ignorespaces
    }
}


\newenvironmentx{proposition}[2][\empty] {

    \newcommand{\Title}{Proposition}

    \ifthenelse{ \equal{#2}{\empty} }{
        % Only one argument supplied, don't need parantheses.
        \par\addvspace{\topsep}
        \noindent\textbf{\Title\  #1}.
        \ignorespaces
    }{
        % Two arguments supplied, show in parantheses.
        \par\addvspace{\topsep}
        \noindent\textbf{\Title\  #1} (#2).
        \ignorespaces
    }
}

\newenvironmentx{corollary}[2][\empty] {

    \newcommand{\Title}{Corollary}

    \ifthenelse{ \equal{#2}{\empty} }{
        % Only one argument supplied, don't need parantheses.
        \par\addvspace{\topsep}
        \noindent\textbf{\Title\  #1}.
        \ignorespaces
    }{
        % Two arguments supplied, show in parantheses.
        \par\addvspace{\topsep}
        \noindent\textbf{\Title\  #1} (#2).
        \ignorespaces
    }
}

\newenvironmentx{remark}[2][\empty] {

    \newcommand{\Title}{Remark}

    \ifthenelse{ \equal{#2}{\empty} }{
        % Only one argument supplied, don't need parantheses.
        \par\addvspace{\topsep}
        \noindent\textbf{\Title\  #1}.
        \ignorespaces
    }{
        % Two arguments supplied, show in parantheses.
        \par\addvspace{\topsep}
        \noindent\textbf{\Title\  #1} (#2).
        \ignorespaces
    }
}

\newenvironmentx{example}[2][\empty] {

    \newcommand{\Title}{Example}

    \ifthenelse{ \equal{#2}{\empty} }{
        % Only one argument supplied, don't need parantheses.
        \par\addvspace{\topsep}
        \noindent\textbf{\Title\  #1}.
        \ignorespaces
    }{
        % Two arguments supplied, show in parantheses.
        \par\addvspace{\topsep}
        \noindent\textbf{\Title\  #1} (#2).
        \ignorespaces
    }
}

\newenvironmentx{exercise}[2][\empty] {

    \newcommand{\Title}{Exercise}

    \ifthenelse{ \equal{#2}{\empty} }{
        % Only one argument supplied, don't need parantheses.
        \par\addvspace{\topsep}
        \noindent\textbf{\Title\  #1}.
        \ignorespaces
    }{
        % Two arguments supplied, show in parantheses.
        \par\addvspace{\topsep}
        \noindent\textbf{\Title\  #1} (#2).
        \ignorespaces
    }
}

\newenvironmentx{question}[2][\empty] {

    \newcommand{\Title}{Question}

    \ifthenelse{ \equal{#2}{\empty} }{
        % Only one argument supplied, don't need parantheses.
        \par\addvspace{\topsep}
        \noindent\textbf{\Title\  #1}.
        \ignorespaces
    }{
        % Two arguments supplied, show in parantheses.
        \par\addvspace{\topsep}
        \noindent\textbf{\Title\  #1} (#2).
        \ignorespaces
    }
}

%%%%%%%%%%%%%%%%%%%%%%%%%%%%%%%%%%%%%%%%%
% Commands for Mathematical Typesetting %
%%%%%%%%%%%%%%%%%%%%%%%%%%%%%%%%%%%%%%%%%

% Redefine \leq and \geq to something nicer looking:
\renewcommand{\leq}{\leqslant}
\renewcommand{\geq}{\geqslant}

% Inner Product:
\DeclareRobustCommand{\InnerProduct}[2]{
    \ifmmode
        \left( #1,#2 \right)
    \else
        \GenericError{\space\space\space\space}
        {Attempting to use \InnerProduct outside of math mode}
    \fi
}

% Vector Norm: 
\DeclareRobustCommand{\Norm}[1]{
    \ifmmode
        \left\lVert #1 \right\rVert
    \else
        \GenericError{\space\space\space\space}
        {Attempting to use \Norm outside of math mode}
    \fi
}

% Principle Argument of Complex Number:
\DeclareMathOperator{\Arg}{Arg}

% Principle Branch of Logarithm of Complex Number:
\DeclareMathOperator{\Log}{Log}

% Image of Function:
\DeclareMathOperator{\im}{im}

% Real part of complex number:
\let\Re\relax
\DeclareMathOperator{\Re}{Re}

% Imaginary part of complex number:
\let\Im\relax
\DeclareMathOperator{\Im}{Im}

% Interior of a Curve:
\DeclareMathOperator{\Int}{Int}

% Exterior of a Curve:
\DeclareMathOperator{\Ext}{Ext}

% Residue of Function at Singularity:
\DeclareMathOperator{\Res}{Res}

% Proof Hint:
\newcommand{\Hint}{\textit{Hint: }}

% Sigma-Algebra:
\newcommand{\SigmaAlgebra}{$\sigma$-algebra}

% Indicator Function:
\newcommand{\Indicator}[1]{\mathbf{1}_{#1}}

% Calligraphic F:
\newcommand{\CalF}{\mathcal{F}}

% This sets page margins to .5 inch if using letter paper, and to 1cm
% if using A4 paper. (This probably isn't strictly necessary.)
% If using another size paper, use default 1cm margins.
\ifthenelse{\lengthtest { \paperwidth = 11in}}
    { \geometry{top=.5in,left=.5in,right=.5in,bottom=.5in} }
    {\ifthenelse{ \lengthtest{ \paperwidth = 297mm}}
        {\geometdry{top=1cm,left=1cm,right=1cm,bottom=1cm} }
        {\geometry{top=1cm,left=1cm,right=1cm,bottom=1cm} }
    }

% Turn off header and footer
\pagestyle{empty}

% Redefine section commands to use less space
\makeatletter
\renewcommand{\section}{\@startsection{section}{1}{0mm}%
                                {-1ex plus -.5ex minus -.2ex}%
                                {0.5ex plus .2ex}%x
                                {\normalfont\large\bfseries}}
\renewcommand{\subsection}{\@startsection{subsection}{2}{0mm}%
                                {-1explus -.5ex minus -.2ex}%
                                {0.5ex plus .2ex}%
                                {\normalfont\normalsize\bfseries}}
\renewcommand{\subsubsection}{\@startsection{subsubsection}{3}{0mm}%
                                {-1ex plus -.5ex minus -.2ex}%
                                {1ex plus .2ex}%
                                {\normalfont\small\bfseries}}
\makeatother

% Define BibTeX command
\def\BibTeX{{\rm B\kern-.05em{\sc i\kern-.025em b}\kern-.08em
    T\kern-.1667em\lower.7ex\hbox{E}\kern-.125emX}}

% Don't print section numbers
\setcounter{secnumdepth}{0}


\setlength{\parindent}{0pt}
\setlength{\parskip}{0pt plus 0.5ex}

\begin{document}
\raggedright
\footnotesize
\begin{multicols}{3}


% multicols parameters
% These lengths are set only within the two main columns
%\setlength{\columnseprule}{0.25pt}
\setlength{\premulticols}{1pt}
\setlength{\postmulticols}{1pt}
\setlength{\multicolsep}{1pt}
\setlength{\columnsep}{2pt}
\setlength{\columnseprule}{0.4pt} % For vertical lines separating columns.

\begin{center}
     \Large{Essentials in Analysis \& Probability} \\
     \footnotesize{Sebastian Müksch, v1, 2019/20}
\end{center}

%%%%%%%%%%%%%%%%%
% Basic Notions %
%%%%%%%%%%%%%%%%%

\begin{example}{1.1}{}

    Simplest \SigmaAlgebra:

        \begin{itemize}
            \setlength{\parskip}{0em}
            \item $\{\emptyset, \Omega\}$, \emph{contained in every} \SigmaAlgebra on $\Omega$,
            \item Family of all subsets of $\Omega$, \emph{containing every} \SigmaAlgebra on $\Omega$.
        \end{itemize}

\end{example}

\begin{exercise}{1.1}{}

    Let $\CalF$ be a \SigmaAlgebra. Then $A_n \in \CalF$ for every integer $n \geq 1$ $\Rightarrow \bigcap_{n=1}^{\infty} A_n \in \CalF$.

\end{exercise}

%%%%%%%%%%%%%%%%%%%%%%%%
% Expectation Integral %
%%%%%%%%%%%%%%%%%%%%%%%%

\section{Expectation Integrals}

\begin{exercise}{3.5}{}

    Let $A \in \CalF$ s.t. $\mu(A) = 0$. Then for \emph{any} measurable function $f: \Omega \to \overline{\mathbb{R}}$:
    
        \begin{align*}
            \int_A f \,d\mu = 0.
        \end{align*}

\end{exercise}

\begin{theorem}{3.8}{Monotone Convergence}

    Let $(f_n)_{n=1}^{\infty}$ be increasing sequence of non-negative, measurable functions $f_n: \Omega \to \overline{\mathbb{R}}$, converging to some $f$. Then:

        \begin{align*}
            \int_{\Omega} \lim_{n \to \infty} f_n \,d\mu = \lim_{n \to \infty} \int_{\Omega} f_n \,d\mu
        \end{align*}

\end{theorem}

\begin{proposition}{3.18}{Markov-Chebyshev's Inequality}

    Let $X$ be a \emph{non-negative} R.V., then

        \begin{align*}
            P(X \geq \lambda) \leq \lambda^{-\alpha} E(X^{\alpha}) \quad \forall \lambda > 0, \alpha > 0.
        \end{align*}

\end{proposition}

\begin{remark}{3.3}{}

    Let $(\Omega, \mathcal{F}, \mu)$ be measure space, $f: \Omega \to \overline{\mathbb{R}}$ \emph{non-negative} $\mathcal{F}$-measurable, then

        \begin{align*}
            \mu(f \geq \lambda) \leq \lambda^{-\alpha} \int_{\Omega} f^{\alpha} \,d\mu \quad \forall \lambda > 0, \alpha > 0.
        \end{align*}

\end{remark}

%%%%%%%%%%%%%%%
% Definitions %
%%%%%%%%%%%%%%%

\begin{definition}{1.1}

    Let $\mathcal{F}$ be a family of subsets of set $\Omega$. $\mathcal{F}$ is called a \emph{$\sigma$-algebra} if:

        \begin{itemize}
            \setlength{\parskip}{0em}
            \item \emph{Closed Under Complement}: $A \in \mathcal{F} \Rightarrow A^c \in \mathcal{F}$,
            \item \emph{Closed Under Arbitrary Union}: $A_n \in \mathcal{F}$ for integer $n \geq 1$ $\Rightarrow \bigcup_{n=1}^{\infty}A_n \in \mathcal{F}$,
            \item \emph{Contains Entire Set}: $\Omega \in \mathcal{F}$
        \end{itemize}

\end{definition}

\begin{definition}{1.2}
    Let $\mathcal{C}$ be a family of subsets of $\Omega$. There exists a $\sigma$-algebra which contains $\mathcal{C}$ \emph{and} which is contained in every $\sigma$-algebra that contains $\mathcal{C}$ (take intersection of all $\sigma$-algebras. Such $\sigma$-algebra is \emph{unique} and called \emph{smallest $\sigma$-algebra containing $\mathcal{C}$} or \emph{$\sigma$-algebra generated by $\mathcal{C}$}, denoted by $\sigma(\mathcal{C})$. Simplest example, let $A \subseteq \Omega$:

        \begin{align*}
            \sigma(A) = \{\emptyset, A, A^c, \Omega\}.
        \end{align*}
\end{definition}

\begin{definition}{2.1.1}

    Let $A \subseteq \Omega$ and $\Indicator{A}$ be defined as follows:

        \begin{align*}
            \Indicator{A}(\omega) =
            \begin{cases}
                1, & \omega \in A \\
                0, & \omega \not\in A
            \end{cases}.
        \end{align*}

    Then $\Indicator{A}$ is a R.V. and called the \emph{indicator (function) of (events) A}.

\end{definition}

\end{multicols}

\end{document}
